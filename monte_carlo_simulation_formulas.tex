
\documentclass{article}
\usepackage{amsmath}
\usepackage{amssymb}

\begin{document}

\title{LaTeX Formulas for Monte Carlo Simulation of Stock Prices and Option Valuation}
\maketitle

Here are the LaTeX representations of all the formulas mentioned in the discussion:

\begin{enumerate}
\item The formula for the stock price simulation:

\begin{equation*}
S_t = S_{t-1} + \mu S_{t-1}\Delta t + \sigma S_{t-1}\epsilon \sqrt{\Delta t}
\end{equation*}

Where:
\begin{itemize}
\item \(S_t\) is the stock price at time \(t\),
\item \(S_{t-1}\) is the stock price at time \(t-1\),
\item \(\mu\) is the drift rate of the stock (which we set equal to the risk-free rate \(r\)),
\item \(\sigma\) is the annual volatility of the stock,
\item \(\Delta t\) is the size of a time step (we used \(\Delta t = 1/252\) for a year of 252 trading days), and
\item \(\epsilon\) is a random shock sampled from the standard normal distribution.
\end{itemize}

\item The formula for the expected payoff of a call option:

\begin{equation*}
\exp(-rT) \times \mathbb{E}[\max(S_T - X, 0)]
\end{equation*}

Where:
\begin{itemize}
\item \(S_T\) is the stock price at maturity,
\item \(X\) is the strike price (which we set equal to the initial stock price \(S_0\)),
\item \(\mathbb{E}[\cdot]\) denotes the expected value (or mean), and
\item The \(\max(a, b)\) function returns the greater of \(a\) and \(b\).
\end{itemize}

\item The Black-Scholes formula for a European call option:

\begin{equation*}
C = S_0 e^{-qT} N(d1) - X e^{-rT} N(d2)
\end{equation*}

Where:
\begin{itemize}
\item \(C\) is the price of the call option,
\item \(S_0\) is the initial stock price,
\item \(r\) is the annual risk-free interest rate,
\item \(T\) is the time to maturity,
\item \(N(\cdot)\) represents the cumulative distribution function of the standard normal distribution,
\item \(q\) is the continuous dividend yield (which we assumed to be 0 as it wasn't provided),
\item \(X\) is the strike price, and
\item \(d1\) and \(d2\) are given by

\begin{align*}
d1 = \frac{\ln \left(\frac{S_0}{X}\right) + \left(r - q + \frac{\sigma^2}{2}\right)T}{\sigma \sqrt{T}}, \\
d2 = d1 - \sigma \sqrt{T}
\end{align*}
\end{itemize}

\item The formula for the stock price in the exact solution for Geometric Brownian Motion (GBM):

\begin{equation*}
S_t = S_0 e^{(r - 0.5\sigma^2)t + \sigma\epsilon\sqrt{t}}
\end{equation*}

Where:
\begin{itemize}
\item \(S_t\) is the stock price at time \(t\),
\item \(S_0\) is the initial stock price,
\item \(r\) is the annual risk-free interest rate,
\item \(\sigma\) is the annual volatility of the stock,
\item \(t\) is the time, and
\item \(\epsilon\) is a standard normal random variable.
\end{itemize}

\item The payoff at maturity for an Asian call option:

\begin{equation*}
\max(\bar{S_T} - X, 0)
\end{equation*}

Where:
\begin{itemize}
\item \(\bar{S_T}\) is the average price of the underlying asset over the life of the option,
\item \(X\) is the strike price, and
\item The \(\max(a, b)\) function returns the greater of \(a\) and \(b\).
\end{itemize}
\end{enumerate}

\end{document}
